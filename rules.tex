\documentclass[10pt,letterpaper]{article}
\include{settings}

%% Include all macros below
\newcommand{\fixme}[1]{\textsc{\textbf{FIXME: {#1}}}}
\newcommand{\withurl}[2]{{#1}\footnote{\texttt{#2}}}
\newcommand{\rulemajor}[1]{\section{#1}}
\begin{document}
\vspace*{0.2in}

\begin{flushleft}
{\Large
\textbf\newline{Ten Simple Rules for Creating an Effective Lesson}
}
\newline
\\
{Greg~Wilson}\textsuperscript{1,*}
\\
\textbf{1} RStudio, Inc. / greg.wilson@rstudio.com
\\
\bigskip
* Corresponding author.
\end{flushleft}

\section*{Abstract}

\fixme{abstract}

\section*{Author Summary}

\fixme{author summary}

\section*{Introduction}

There are many kinds of lessons, both formal and informal, from seconds long to lifelong.
Most people have sat (or suffered) through hundreds of these,
but have never been shown how to design ones that are effective.
These ten simple rules for creating lessons are:

\begin{itemize}

\item based on current educational research \cite{Nuth2007,Ambr2010,Brow2018},

\item filtered by what can be done by non-specialists
  with limited time and resources \cite{Hust2012,Lang2016,Rice2018},
  and

\item prioritized by experience teaching and training people to teach together
  \cite{Deve2018,Wils2016,Wils2018}.

\end{itemize}

\section*{A Little Bit of Theory}

\fixme{summarize theory}

\rulemajor{Use Learner Personas to Define Your Audience}

The first step in creating a good lesson is figuring out who the audience is.
One way to do this is to make up biographies of two or three target learners.
This technique is borrowed from user interface designers, who create short
profiles of typical users to help them think about their audience, and the
profiles themselves are called \emph{personas}.

Learner personas have five parts:

\begin{enumerate}

\item the person's general background,

\item what they already know,

\item what \emph{they} think they want to do (as opposed to what someone who
  already understands the subject thinks),

\item how the lesson will help them, and

\item any special needs they might have.

\end{enumerate}

A learner persona for a weekend introduction to programming aimed at college
students might be:

\begin{enumerate}

\item Jorge has just moved from Costa Rica to Canada to study agricultural
  engineering. He has joined the college soccer team, and is looking forward to
  learning how to play ice hockey.

\item Other than using Excel, Word, and the Internet, Jorge's most significant
  previous experience with computers is helping his sister build a WordPress
  site for the family business back home in Costa Rica.

\item Jorge needs to measure properties of soil from nearby farms using a
  handheld device that sends logs in a text format to his computer.  Right now,
  Jorge has to open each file in Excel, crop the first and last points, and
  calculate an average.

\item This workshop will show Jorge how to write a little Python program to read
  the data, select the right values from each file, and calculate the required
  statistics.

\item Jorge can read English well, but still struggles sometimes to keep up with
  spoken conversation (especially if it involves a lot of new jargon).

\end{enumerate}
  
Rather than writing new personas for every lesson or course, instructors often
create and share a handful that cover everyone they hope to teach, then pick a
few from that set to describe who particular material is intended for.  Used
this way, personas become a convenient shorthand for design issues: when
speaking with each other, teachers can say, "Would Jorge understand why we're
doing this?" or, "What installation problems would Jorge face?"

Personas help you remember one of the most important rules of teaching:
\emph{you are not your learners}.  The people you teach will almost always have
different backgrounds, different capabilities, and different ambitions than you;
personas help you keep your lessons focused on what they need rather than on
what your younger self might have wanted.

\rulemajor{Write Summative Assessments to Set Concrete Goals}

\emph{Summative assessment} is something done at the end of a lesson to tell
whether the desired learning has taken place: a driving test, performance of a
piece of music, a written examination, or something else of that kind.
Summative assessments are usually used as gates (e.g., ``Is it now safe for this
person to drive on their own?''), but they are also a good way to clarify the
learning objectives for a lesson.  ``Understand linear regression'' is
hopelessly vague; a much better way to set the goal for a lesson is:

\begin{quotation}

  Write a short R script that reads the tabular data in \texttt{housing.csv}
  and uses the \texttt{lm} function to calculate a regression coefficient
  relating house price to purchaser age.

\end{quotation}

This is better because it gives the lesson author a concrete goal to work
toward: nothing goes in the lesson except what is needed to complete the
summative assesments.  This helps reduce content bloat, and also tells the
author when the lesson is done.

Writing summative assessments early in the lesson design process also helps
ensure that outcomes are actually checkable.  Since telepathy is not yet widely
available, it is impossible for instructors to know what learners do and don't
understand.  Instead, we must ask them to demonstrate that they're able to do
something that they couldn't do without the desired understanding.

Finally, creating summative assessments early can help authors stay connected to
their learners' goals.  Each summative assessment should embody an
\emph{authentic task}, i.e., something that an actual learner actually wants to
do.  Continuing with the statistical example above, calculating a regression
coefficient may be an authentic task for someone who already knows enough
statistics to understand what such coefficients are good for.  If the intended
learners are not yet that experienced (which you will know from your personas),
this exercise could be extended to have them make some sort of judgment based
on the coefficient in order to answer the deeper question, ``Why bother?''

\rulemajor{Write Formative Assessments for Pacing, Design, Preparation, and Reinforcement}

The counterpoint to summative assessment is \emph{formative assessment}, which is
checks used while learning is taking place to form (or shape) the teaching.
Asking learners for questions is a common, but relatively ineffective, kind of
formative assessment.  What works better is to have them solve a short
problem---one that can be done in 1--2 minutes so as not to derail the flow of
the lesson.

Checking in with learners this way every 10--15 minutes accomplishes several
things:

\begin{description}

\item[Pacing:] Asking, ``Does everyone understand?''  almost always produces
  false positives.  In contrast, if any substantial fraction of your learners
  cannot do a formative assessment correctly, you know right then and there that
  you need to re-explain the most recent material.  When you start doing this,
  you will feel like you're going more slowly, but that's because you will now
  be teaching at the speed at which your audience can learn rather than the
  speed at which you can talk.

\item[Design:] Creating formative assessments that build toward a lesson's
  summative assessment gives you a structure for your lesson.  Returning to the
  regression example, the summative assessment tells you that you should have
  exercises along the way in which learners load CSV data, use the \texttt{lm}
  function with appropriate parameters, and interpret the result.  Writing a few
  minutes of material for each of these subjects is less intimidating than
  trying to explain the whole topic at once.
  
\item[Preparation:] Formative assessments give learners practice with the
  concepts, methods, and tools they will use when doing the lesson's summative
  assessment, and tells them where to focus their revision.  Switching from
  statistics to music, if a violinist is able to do the bowing and fingering
  exercises for a piece, but is struggling with the rhythmic patterns, that
  tells her where she should spend her study time.
  
\item[Reinforcement:] Learners remember things better if they use material right
  away, and having formative assessments during the lesson does this.

\item[Scope:] Breaking a summative assessment down into parts and creating
  formative assessments for each usually shows you that you are trying to cram
  too much into one lesson.  Writing assessments is therefore iterative, as
  early drafts of summative assessments are re-scoped to only require as much
  material as can plausibly be covered.

\end{description}
  
\rulemajor{Design for Peer Instruction}

No matter how good a teacher is, she can only say one thing at a time.  How then
can she clear up many different misconceptions in a reasonable time? The best
solution developed so far is \emph{peer instruction}. Originally created by Eric
Mazur at Harvard \cite{Crou2001}, it has been studied extensively in a wide
variety of contexts (e.g., \cite{Vick2015,Port2016}).

Peer instruction is essentially a scalable way to provide one-to-one mentorship.
It interleaves formative assessment with student discussion as follows:

\begin{enumerate}

\item Give a brief introduction to the topic, either in class or in out-of-class
  reading.

\item Give learners a multiple choice question (MCQ).

\item Have all the students vote on their answers to the MCQ.

  \begin{enumerate}

   \item If the students all have the right answer, move on.

   \item If they all have the same wrong answer, address that specific
     misconception.

   \item If they have a mix of right and wrong answers, give them several
     minutes to discuss those answers with one another in small groups
     (typically 2--4 students) and then reconvene and vote again.

  \end{enumerate}

\end{enumerate}

The questions posed to learners don't have to be MCQs: matching terms to
definitions can be equally effective, as can Parsons Problems (in which they are
given the jumbled parts of a solution and must put them in the right order
\cite{Pars2006}).  Whatever mix is used, the lesson must build toward them, and
the question must probe for conceptual understanding and misconceptions (rather
than check simple factual knowledge).

Group discussion significantly improves students' understanding because it
forces them to clarify their thinking, which can be enough to call out gaps in
reasoning.  Re-polling the class then lets the teacher know if they can move on,
or if further explanation is necessary. A final round of additional explanation
and discussion after the correct answer is presented gives students one more
chance to solidify their understanding.

But could this be a false positive? Are results improving because of increased
understanding during discussion, or simply from a follow-the-leader effect?
\cite{Smit2009} tested this by following the first question with a second one
that students answer individually and found that peer discussion actually does
enhance understanding, even when none of the students in a discussion group
originally knew the correct answer.

It is important to have learners vote publicly so that they can't change their
minds afterward and rationalize it by making excuses to themselves like ``I just
misread the question''.  Some of the value of peer instruction comes from having
their answer be wrong and having to think through the reasons why.  This is
called the \emph{hypercorrection effect} \cite{Metc2016}. Most people don't like
to be told they're wrong, so it's reasonable to assume that the more confident
someone is that the answer they've given in a test is correct, the harder it is
to change their mind if they were actually wrong.  However, it turns out that
the opposite is true: the more confident someone is that they were right, the
more likely they are not to repeat the error if they are corrected.

\rulemajor{Use Worked Examples}

\fixme{worked examples}

\rulemajor{Use Concreteness Fading}

\fixme{concreteness fading}

\rulemajor{Show How to Detect, Diagnose, and Correct Common Mistakes}

\fixme{common mistakes}

\rulemajor{Eliminate Incentives for Cheating}

\fixme{cheating}

\rulemajor{Be Inclusive}

\fixme{inclusivity}

\rulemajor{Encourage Effective Learning Strategies}

Some learning strategies are provably more effective than others \cite{Rohr2015,Kang2016,Miya2018},
so lessons should be designed to encourage their use.
As summarized in \cite{Wein2018a,Wein2018b},
the six most important are:

\begin{description}

\item[Spaced Practice:] Ten hours of study spread out over five days is more
  effective than two five-hour days, and far better than one ten-hour day. You
  should therefore create lessons and exercises that include some older material
  in each new lesson.  According to \cite{Mill2016}, ``The lectures that
  predominate in face-to-face courses are relatively ineffective ways to teach,
  but they probably contribute to spacing material over time, because they
  unfold in a set schedule over time.  In contrast, depending on how the courses
  are set up, online students can sometimes avoid exposure to material
  altogether until an assignment is nigh.''

\item[Retrieval Practice:] Researchers now believe that the limiting factor for
  long-term memory is not retention (what is stored), but recall (what can be
  accessed).  Recall of specific information improves with practice, so outcomes
  in real situations can be improved by taking practice tests or summarizing the
  details of a topic from memory and then checking what was and wasn't
  remembered. For example, \cite{Karp2008} found that repeated testing improved
  recall of word lists from 35\% to 80\%.

\item[Interleaving:] One way to space retrieval practice is to interleave study
  of different topics: instead of mastering one subject, then the next, then a
  third, shuffle the order. Even better, switch up the order: A-B-C-B-A-C is
  better than A-B-C-A-B-C, which in turn is better than A-A-B-B-C-C
  \cite{Rohr2015}. This is effective because interleaving fosters creation of
  more links between different topics, which in turn increases retention and
  recall.

\item[Elaboration:] Having learners explain things to themselves as they go
  along improves understanding and recall. One way to do this is to follow up
  each answer on a practice quiz with an explanation of why that answer is
  correct, or conversely with an explanation of why some other plausible answer
  isn't. Another is to have learners explain how a new idea is similar to or
  different from one they have seen previously.

\item[Concrete Examples:] One specific form of elaboration that is useful enough
  to deserve its own heading is the use of concrete examples.  As discussed in
  the rule on concreteness fading, every statement of a general principle should
  be accompanied by one or more examples of its use, or conversely take each
  particular problem and list the general principles it
  embodies. \cite{Raws2014} found that interleaving examples and definitions
  made it more likely that learners would remember the latter correctly.

  Another approach is to teach by contrast, i.e., to show learners what a
  solution is \emph{not}, or what kind of problem a technique \emph{won't}
  solve. For example, when showing children how to simplify fractions, it's
  important to give them a few like 5/7 that can't be simplified so that they
  don't become frustrated looking for answers that don't exist.

\item[Dual Coding:] Different subsystems in our brains handle and store
  linguistic and visual information, and if complementary information is
  presented through both channels, then they can reinforce one another. However,
  learning is more effective when the same information is \emph{not} presented
  simultaneously in two different channels \cite{Maye2003,Maye2009}, because then
  the brain has to expend effort to check the channels against each other.

\end{description}

\section*{Conclusion}

\fixme{Conclusion}

\bibliography{rules}

\end{document}
