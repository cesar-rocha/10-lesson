\documentclass[10pt,letterpaper]{article}
\include{settings}

%% Include all macros below
\newcommand{\fixme}[1]{\textsc{\textbf{FIXME: {#1}}}}
\newcommand{\withurl}[2]{{#1}\footnote{\texttt{#2}}}
\newcommand{\rulemajor}[1]{\section{#1}}
\begin{document}
\vspace*{0.2in}

\begin{flushleft}
{\Large
\textbf\newline{Ten Simple Rules for Creating an Effective Lesson}
}
\newline
\\
{Greg~Wilson}\textsuperscript{1,*}
\\
\textbf{1} RStudio, Inc. / greg.wilson@rstudio.com
\\
\bigskip
* Corresponding author.
\end{flushleft}

\section*{Abstract}

\fixme{abstract}

\section*{Author Summary}

\fixme{author summary}

\section*{Introduction}

There are many kinds of lessons, both formal and informal, from seconds long to lifelong.
Most people have sat (or suffered) through hundreds of these,
but have never been shown how to design ones that are effective.
These ten simple rules for creating lessons are:

\begin{itemize}

\item based on current educational research \cite{Nuth2007,Ambr2010,Brow2018},

\item filtered by what can be done by non-specialists
  with limited time and resources \cite{Hust2012,Lang2016,Rice2018},
  and

\item prioritized by experience teaching and training people to teach together
  \cite{Deve2018,Wils2016,Wils2018}.

\end{itemize}

\section*{A Little Bit of Theory}

\fixme{summarize theory}

\rulemajor{Use Learner Personas to Define Your Audience}

\fixme{learner personas}

\rulemajor{Write Summative Assessments to Set Concrete Goals}

\fixme{summative assessment}

\rulemajor{Write Formative Assessments for Pacing, Design, Preparation, and Reinforcement}

\fixme{formative assessment}

\rulemajor{Design for Peer Instruction}

\fixme{peer instruction}

\rulemajor{Use Worked Examples}

\fixme{worked examples}

\rulemajor{Use Concreteness Fading}

\fixme{concreteness fading}

\rulemajor{Show How to Detect, Diagnose, and Correct Common Mistakes}

\fixme{common mistakes}

\rulemajor{Eliminate Incentives for Cheating}

\fixme{cheating}

\rulemajor{Be Inclusive}

\fixme{inclusivity}

\rulemajor{Encourage Effective Learning Strategies}

Some learning strategies are provably more effective than others \cite{Rohr2015,Kang2016,Miya2018},
so lessons should be designed to encourage their use.
As summarized in \cite{Wein2018a,Wein2018b},
the six most important are:

\begin{description}

\item[Spaced Practice:] Ten hours of study spread out over five days is more
  effective than two five-hour days, and far better than one ten-hour day. You
  should therefore create lessons and exercises that include some older material
  in each new lesson.  According to \cite{Mill2016}, ``The lectures that
  predominate in face-to-face courses are relatively ineffective ways to teach,
  but they probably contribute to spacing material over time, because they
  unfold in a set schedule over time.  In contrast, depending on how the courses
  are set up, online students can sometimes avoid exposure to material
  altogether until an assignment is nigh.''

\item[Retrieval Practice:] Researchers now believe that the limiting factor for
  long-term memory is not retention (what is stored), but recall (what can be
  accessed).  Recall of specific information improves with practice, so outcomes
  in real situations can be improved by taking practice tests or summarizing the
  details of a topic from memory and then checking what was and wasn't
  remembered. For example, \cite{Karp2008} found that repeated testing improved
  recall of word lists from 35\% to 80\%.

  One powerful finding in learning research is the \emph{hypercorrection effect}
  \cite{Metc2016}. Most people don't like to be told they're wrong, so it's
  reasonable to assume that the more confident someone is that the answer
  they've given in a test is correct, the harder it is to change their mind if
  they were actually wrong. However, it turns out that the opposite is true: the
  more confident someone is that they were right, the more likely they are not
  to repeat the error if they are corrected.

\item[Interleaving:] One way to space retrieval practice is to interleave study
  of different topics: instead of mastering one subject, then the next, then a
  third, shuffle the order. Even better, switch up the order: A-B-C-B-A-C is
  better than A-B-C-A-B-C, which in turn is better than A-A-B-B-C-C
  \cite{Rohr2015}. This is effective because interleaving fosters creation of
  more links between different topics, which in turn increases retention and
  recall.

\item[Elaboration:] Having learners explain things to themselves as they go
  along improves understanding and recall. One way to do this is to follow up
  each answer on a practice quiz with an explanation of why that answer is
  correct, or conversely with an explanation of why some other plausible answer
  isn't. Another is to have learners explain how a new idea is similar to or
  different from one they have seen previously.

\item[Concrete Examples:] One specific form of elaboration that is useful enough
  to deserve its own heading is the use of concrete examples.  As discussed in
  the rule on concreteness fading, every statement of a general principle should
  be accompanied by one or more examples of its use, or conversely take each
  particular problem and list the general principles it
  embodies. \cite{Raws2014} found that interleaving examples and definitions
  made it more likely that learners would remember the latter correctly.

  Another approach is to teach by contrast, i.e., to show learners what a
  solution is \emph{not}, or what kind of problem a technique \emph{won't}
  solve. For example, when showing children how to simplify fractions, it's
  important to give them a few like 5/7 that can't be simplified so that they
  don't become frustrated looking for answers that don't exist.

\item[Dual Coding:] Different subsystems in our brains handle and store
  linguistic and visual information, and if complementary information is
  presented through both channels, then they can reinforce one another. However,
  learning is more effective when the same information is \emph{not} presented
  simultaneously in two different channels \cite{Maye2003,Maye2009}, because then
  the brain has to expend effort to check the channels against each other.

\end{description}

\section*{Conclusion}

\fixme{Conclusion}

\bibliography{rules}

\end{document}
