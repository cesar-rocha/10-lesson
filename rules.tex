\documentclass[10pt,letterpaper]{article}
\include{settings}

%% Define per-paper macros.
\newcommand{\withurl}[2]{{#1}\footnote{\texttt{#2}}}
\newcommand{\rulemajor}[1]{\section{#1}}
\begin{document}
\vspace*{0.2in}

\begin{flushleft}
{\Large
\textbf\newline{Ten Simple Rules for Creating an Effective Lesson}
}
\newline
\\
{Greg~Wilson}\textsuperscript{1,*}
\\
\textbf{1} RStudio, Inc. / greg.wilson@rstudio.com
\\
\bigskip
* Corresponding author.
\end{flushleft}

\section*{Abstract}

We present ten rules for building effective lessons that are grounded in
empirical research on pedagogy and cognitive psychology, and which we have found
to be practically useful in both classroom and free-range settings.

\section*{Author Summary}

As a species, we know as much about teaching and learning as we do about public
health, but most people who teach at the post-secondary level are never
introduced to even the basics of evidence-based pedagogy.  Knowing just a few
key facts will help you build more effective lessons in less time and with less
pain, and will also make those lessons easier for your peers to find and re-use.
This paper presents ten rules that you can apply immediately and explains why
they work.

\section*{Introduction}

There are many kinds of lessons, both formal and informal, from seconds long to
lifelong.  Most people have sat (or suffered) through hundreds of these, but
have never been shown how to design ones that are effective.  These ten simple
rules for creating lessons are:

\begin{itemize}

\item based on current educational research \cite{Nuth2007,Ambr2010,DeBr2015,Dida2016,Brow2018,Mark2018},

\item filtered by what can be done by non-specialists with limited time and
  resources \cite{Hust2012,Lang2016}, and

\item prioritized by experience teaching and training people to teach together
  \cite{Deve2018,Wils2016,Wils2018}.

\end{itemize}

The key insight that underpins all of these rules is that \emph{learning is both
  a cognitive and a social activity}.  On the cognitive side, we can model
learning as shown in Figure~\ref{cognitive-model}.

\begin{figure}[ht]
\includegraphics[width=12cm]{cognitive-model.pdf}
\caption{A Cognitive Model of Learning (adapted from \cite{Maye2009}).}
\label{cognitive-model}
\end{figure}

Incoming information (the lesson) passes through a \emph{sensory register} that
has physically separate channels for visual and auditory information and is
stored in \emph{short-term memory}, where it is used to construct a \emph{verbal
  model} (sometimes also called a \emph{linguistic model}) and a separate
\emph{visual model}.  These are then integrated and stored in \emph{long-term
  memory} as facts and relationships.  If those facts and relationships are
strengthened by use, they can later be recalled and applied, and we say that
learning has taken place.

One key feature of this model is that short-term memory is very limited:
\cite{Mill1956} famously estimated its size as $7{\pm}2$ items, and more
recent studies place the figure closer to 4.  If too much information is
presented too quickly, material spills out of short-term memory before it can be
integrated and stored, and learning does not occur.

A second key feature is that the brain's processing power is also very limited.
Effort spent identifying key facts or reconciling the linguistic and visual
input streams reduces the power available for organizing new information and
connecting it to what's already present.

Learning is also a social activity.  In \cite{Litt2004}, for example, Kenneth
Wesson wrote, ``If poor inner-city children consistently outscored children from
wealthy suburban homes on standardized tests, is anyone naive enough to believe
that we would still insist on using these tests as indicators of success?''
Learners who feel motivated will learn more; learners who feel that they may
not be judged on their merits, or who have experienced unequal treatment in
the past, will learn less, and lesson designers must take this into account
if they are to create effective lessons.

\rulemajor{Use Learner Personas to Define Your Audience}

The first step in creating a good lesson is figuring out who the audience is.
One way to do this is to make up biographies of two or three target learners.
This technique is borrowed from user interface designers, who create short
profiles of typical users to help them think about their audience, and the
profiles themselves are called \emph{personas}.

Learner personas have five parts:

\begin{enumerate}

\item the person's general background,

\item what they already know,

\item what \emph{they} think they want to do (as opposed to what someone who
  already understands the subject thinks),

\item how the lesson will help them, and

\item any special needs they might have.

\end{enumerate}

A learner persona for a weekend introduction to programming aimed at college
students might be:

\begin{enumerate}

\item Jorge has just moved from Costa Rica to Canada to study agricultural
  engineering. He has joined the college soccer team, and is looking forward to
  learning how to play ice hockey.

\item Other than using Excel, Word, and the Internet, Jorge's most significant
  previous experience with computers is helping his sister build a WordPress
  site for the family business back home in Costa Rica.

\item Jorge needs to measure properties of soil from nearby farms using a
  handheld device that sends logs in a text format to his computer.  Right now,
  Jorge has to open each file in Excel, crop the first and last points, and
  calculate an average.

\item This workshop will show Jorge how to write a little Python program to read
  the data, select the right values from each file, and calculate the required
  statistics.

\item Jorge can read English well, but still struggles sometimes to keep up with
  spoken conversation (especially if it involves a lot of new jargon).

\end{enumerate}

Rather than writing new personas for every lesson or course, instructors often
create and share a handful that cover everyone they hope to teach, then pick a
few from that set to describe who particular material is intended for.  Used
this way, personas become a convenient shorthand for design issues: when
speaking with each other, teachers can say, "Would Jorge understand why we're
doing this?" or, "What installation problems would Jorge face?"

Personas help you remember one of the most important rules of teaching:
\emph{you are not your learners}.  The people you teach will almost always have
different backgrounds, different capabilities, and different ambitions than you;
personas help you keep your lessons focused on what they need rather than on
what your younger self might have wanted.

\rulemajor{Design for Effective Learning Strategies}

Some learning strategies are provably more effective than others
\cite{Rohr2015,Kang2016,Miya2018}, so lessons should be designed to encourage
their use.  As summarized in \cite{Wein2018a,Wein2018b}, the six most important
are:

\begin{description}

\item[Spaced Practice:] Ten hours of study spread out over five days is more
  effective than two five-hour days, and far better than one ten-hour day. You
  should therefore create lessons and exercises that include some older material
  in each new lesson.  According to \cite{Mill2016}, ``The lectures that
  predominate in face-to-face courses are relatively ineffective ways to teach,
  but they probably contribute to spacing material over time, because they
  unfold in a set schedule over time.  In contrast, depending on how the courses
  are set up, online students can sometimes avoid exposure to material
  altogether until an assignment is nigh.''

\item[Retrieval Practice:] Researchers now believe that the limiting factor for
  long-term memory is not retention (what is stored), but recall (what can be
  accessed).  Recall of specific information improves with practice, so outcomes
  in real situations can be improved by taking practice tests or summarizing the
  details of a topic from memory and then checking what was and wasn't
  remembered. For example, \cite{Karp2008} found that repeated testing improved
  recall of word lists from 35\% to 80\%.

\item[Interleaving:] One way to space retrieval practice is to interleave study
  of different topics: instead of mastering one subject, then the next, then a
  third, shuffle the order. Even better, switch up the order: A-B-C-B-A-C is
  better than A-B-C-A-B-C, which in turn is better than A-A-B-B-C-C
  \cite{Rohr2015}. This is effective because interleaving fosters creation of
  more links between different topics, which in turn increases retention and
  recall.

\item[Elaboration:] Having learners explain things to themselves as they go
  along improves understanding and recall. One way to do this is to follow up
  each answer on a practice quiz with an explanation of why that answer is
  correct, or conversely with an explanation of why some other plausible answer
  isn't. Another is to have learners explain how a new idea is similar to or
  different from one they have seen previously.

\item[Concrete Examples:] One specific form of elaboration that is useful enough
  to deserve its own heading is the use of concrete examples.  As discussed in
  the rule on concreteness fading, every statement of a general principle should
  be accompanied by one or more examples of its use, or conversely take each
  particular problem and list the general principles it
  embodies. \cite{Raws2014} found that interleaving examples and definitions
  made it more likely that learners would remember the latter correctly.

  Another approach is to teach by contrast, i.e., to show learners what a
  solution is \emph{not}, or what kind of problem a technique \emph{won't}
  solve. For example, when showing children how to simplify fractions, it's
  important to give them a few like 5/7 that can't be simplified so that they
  don't become frustrated looking for answers that don't exist.

\item[Dual Coding:] Different subsystems in our brains handle and store
  linguistic and visual information, and if complementary information is
  presented through both channels, then they can reinforce one another. However,
  learning is more effective when the same information is \emph{not} presented
  simultaneously in two different channels \cite{Maye2003,Maye2009}, because
  then the brain has to expend effort to check the channels against each other.

\end{description}

\rulemajor{Write Summative Assessments to Set Concrete Goals}

\emph{Summative assessment} is something done at the end of a lesson to tell
whether the desired learning has taken place: a driving test, performance of a
piece of music, a written examination, or something else of that kind.
Summative assessments are usually used as gates (e.g., ``Is it now safe for this
person to drive on their own?''), but they are also a good way to clarify the
learning objectives for a lesson.  ``Understand linear regression'' is
hopelessly vague; a much better way to set the goal for a lesson is:

\begin{quotation}

  \noindent
  Write a short R script that reads the tabular data in \texttt{housing.csv}
  and uses the \texttt{lm} function to calculate a regression coefficient
  relating house price to purchaser age.

\end{quotation}

This is better because it gives the lesson author a concrete goal to work
toward: nothing goes in the lesson except what is needed to complete the
summative assessments.  This helps reduce content bloat, and also tells the
author when the lesson is done.

Writing summative assessments early in the lesson design process also helps
ensure that outcomes are actually checkable.  Since telepathy is not yet widely
available, it is impossible for instructors to know what learners do and don't
understand.  Instead, we must ask them to demonstrate that they're able to do
something that they couldn't do without the desired understanding.

Finally, creating summative assessments early can help authors stay connected to
their learners' goals.  Each summative assessment should embody an
\emph{authentic task}, i.e., something that an actual learner actually wants to
do.  Early on, authentic tasks should be learners' own goals; as they advance
and are able to make sense of generalizations, these tasks may be extensions or
generalizations of earlier solutions.

Continuing with the statistical example above, calculating a regression
coefficient may be an authentic task for someone who already knows enough
statistics to understand what such coefficients are good for.  If the intended
learners are not yet that experienced (which you will know from your personas),
this exercise could be extended to have them make some sort of judgment based
on the coefficient in order to answer the deeper question, ``Why bother?''

\rulemajor{Write Formative Assessments for Pacing, Design, Preparation, and Reinforcement}

The counterpoint to summative assessment is \emph{formative assessment}, which
is checks used while learning is taking place to form (or shape) the teaching.
Asking learners for questions is a common, but relatively ineffective, kind of
formative assessment.  What works better is to give them a short problem---one
that can be done in 1--2 minutes so as not to derail the flow of the lesson, and
which will help them uncover and confront their misconceptions about the topic
being taught.

Checking in with learners this way every 10--15 minutes accomplishes several
things:

\begin{description}

\item[Pacing:] Asking, ``Does everyone understand?''  almost always produces
  false positives.  In contrast, if any substantial fraction of your learners
  cannot do a formative assessment correctly, you know right then and there that
  you need to re-explain the most recent material.  When you start doing this,
  you will feel like you're going more slowly, but that's because you will now
  be teaching at the speed at which your audience can learn rather than the
  speed at which you can talk.

\item[Design:] Creating formative assessments that build toward a lesson's
  summative assessment gives you a structure for your lesson.  Returning to the
  regression example, the summative assessment tells you that you should have
  exercises along the way in which learners load CSV data, use the \texttt{lm}
  function with appropriate parameters, and interpret the result.  Writing a few
  minutes of material for each of these subjects is less intimidating than
  trying to explain the whole topic at once.
  
\item[Preparation:] Formative assessments give learners practice with the
  concepts, methods, and tools they will use when doing the lesson's summative
  assessment, and tells them where to focus their revision.  Switching from
  statistics to music, if a violinist is able to do the bowing and fingering
  exercises for a piece, but is struggling with the rhythmic patterns, that
  tells her where she should spend her study time.
  
\item[Reinforcement:] Learners remember things better if they use material right
  away, and having formative assessments during the lesson does this.

\item[Scope:] Breaking a summative assessment down into parts and creating
  formative assessments for each usually shows you that you are trying to cram
  too much into one lesson.  Writing assessments is therefore iterative, as
  early drafts of summative assessments are re-scoped to only require as much
  material as can plausibly be covered.

\end{description}

\cite{Broo2016,Majo2015,Rice2018} offer inspiration for a wide variety of
different kinds of summative and formative assessment exercises.

\rulemajor{Integrate Visual and Linguistic Information}

Research by Mayer and colleagues on the split-attention effect is closely
related to cognitive load theory \cite{Maye2003}.  As described in the
introduction, linguistic and visual input are processed by different parts of
the human brain, and linguistic and visual memories are stored separately as
well. This means that correlating linguistic and visual streams of information
takes cognitive effort: when someone reads something while hearing it spoken
aloud, their brain can't help but check that it's getting the same information
on both channels.

Learning is therefore more effective when information is presented
simultaneously in two different channels, but when that information is
complementary rather than redundant. For example, people generally find it
harder to learn from a video that has both narration and on-screen captions than
from one that has either the narration or the captions but not both, because
some of their attention has to be devoted to checking that the narration and the
captions agree with each other. Two notable exceptions to this are people who do
not yet speak the language well and people with hearing exercises or other
special needs, both of whom may find that the extra effort is a net benefit.

This is why it's more effective to draw a diagram piece by piece while teaching
rather than to present the whole thing at once. If parts of the diagram appear
at the same time as things are being said, the two will be correlated in the
learner's memory. Pointing at part of the diagram later is then more likely to
trigger recall of what was being said when that part was being drawn.

The split-attention effect does \emph{not} mean that learners shouldn't try to
reconcile multiple incoming streams of information---after all, this is
something they have to do in the real world \cite{Atki2000}. Instead, it means
that instruction shouldn't require it while people are mastering unit skills;
instead, using multiple sources of information simultaneously should be treated
as a separate learning task.
  
\rulemajor{Design for Peer Instruction}

No matter how good a teacher is, she can only say one thing at a time.  How then
can she clear up many different misconceptions in a reasonable time? The best
solution developed so far is \emph{peer instruction}. Originally created by Eric
Mazur at Harvard \cite{Crou2001}, it has been studied extensively in a wide
variety of contexts (e.g., \cite{Vick2015,Port2016}).

Peer instruction is essentially a scalable way to provide one-to-one mentorship.
It interleaves formative assessment with student discussion as follows:

\begin{enumerate}

\item Give a brief introduction to the topic, either in class or in out-of-class
  reading.

\item Give learners a multiple choice question (MCQ).

\item Have all the students vote on their answers to the MCQ.

  \begin{enumerate}

   \item If the students all have the right answer, move on.

   \item If they all have the same wrong answer, address that specific
     misconception.

   \item If they have a mix of right and wrong answers, give them several
     minutes to discuss those answers with one another in small groups
     (typically 2--4 students) and then reconvene and vote again.

  \end{enumerate}

\end{enumerate}

The questions posed to learners don't have to be MCQs: matching terms to
definitions can be equally effective, as can Parsons Problems (in which they are
given the jumbled parts of a solution and must put them in the right order
\cite{Pars2006}).  Whatever mix is used, the lesson must build toward them, and
the question must probe for conceptual understanding and misconceptions (rather
than check simple factual knowledge).

Group discussion significantly improves students' understanding because it
forces them to clarify their thinking, which can be enough to call out gaps in
reasoning.  Re-polling the class then lets the teacher know if they can move on,
or if further explanation is necessary. A final round of additional explanation
and discussion after the correct answer is presented gives students one more
chance to solidify their understanding.

But could this be a false positive? Are results improving because of increased
understanding during discussion, or simply from a follow-the-leader effect?
\cite{Smit2009} tested this by following the first question with a second one
that students answer individually and found that peer discussion actually does
enhance understanding, even when none of the students in a discussion group
originally knew the correct answer.

It is important to have learners vote publicly so that they can't change their
minds afterwards and rationalize it by making excuses to themselves like ``I just
misread the question''.  Some of the value of peer instruction comes from having
their answer be wrong and having to think through the reasons why.  This is
called the \emph{hypercorrection effect} \cite{Metc2016}. Most people don't like
to be told they're wrong, so it's reasonable to assume that the more confident
someone is that the answer they've given in a test is correct, the harder it is
to change their mind if they were actually wrong.  However, it turns out that
the opposite is true: the more confident someone is that they were right, the
more likely they are not to repeat the error if they are corrected.

\rulemajor{Use Worked Examples and Concreteness Fading}

A worked example is a step-by-step demonstration of how to solve a problem or do
some task.  By giving the steps in order, the instructor reduces the learner's
cognitive load, which accelerates learning \cite{Atki2000,Paas2003}.

However, worked examples become less effective as learners acquire more
expertise \cite{Kaly2003,Kaly2007}, a phenomenon known as the \emph{expertise
  reversal effect}.  In brief, as learners build their own mental models of what
to do and how to do it, the detailed step-by-step breakdown of a worked example
starts to get in the way.  This again is a reason to develop learner personas
before writing lessons, and is also the reason why tutorials and manual pages
both need to exist: what's appropriate for a newcomer is frustrating for an
expert, while what jogs an expert's memory may be incomprehensible to a novice.

One powerful way to use worked examples is to present a series of \emph{faded
  examples} \cite{Schw2009}.  The first example in the series is a complete use
of a problem-solving strategy; each subsequent example gives the learner more
blanks to fill in. The material that isn't blank is often referred to as
\emph{scaffolding}, since it serves the same purpose as the scaffolding set up
temporarily at a building site.

Faded examples can be used in almost every kind of teaching, from sport and
music to contract law. Someone teaching Python programming might use them by
first explaining how to calculate the total length of a list of words:

\begin{verbatim}
# total_length(["red", "green", "blue"]) => 12
define total_length(list_of_words):
    total = 0
    for each word in list_of_words:
        total = total + word.length()
    return total
\end{verbatim}

and then asking learners to fill in the blanks in this (which focuses their
attention on control structures):

\begin{verbatim}
# word_lengths(["red", "green", "blue"]) => [3, 5, 4]
define word_lengths(list_of_words):
    list_of_lengths = []
    for each ____ in ____:
        list_of_lengths.append(____)
    return list_of_lengths
\end{verbatim}

The next problem might be this (which focuses their attention on updating the
final result):

\begin{verbatim}
# join_all(["red", "green", "blue"]) => "redgreenblue"
define join_all(list_of_words):
    joined_words = ____
    for each ____ in ____:
        ____
    return joined_words
\end{verbatim}

Learners would finally be asked to write an entire function on their own:

\begin{verbatim}
# make_acronym(["red", "green", "blue"]) => "RGB"
define make_acronym(list_of_words):
    ____
\end{verbatim}

At each step, learners have one new problem to tackle, which is less
intimidating than a blank screen or a blank sheet of paper.  Faded examples also
encourage learners (and instructors) to think about the similarities and
differences between various approaches.

Worked examples are themselves an example of \emph{concreteness fading}
\cite{Gold2005,Fyfe2014}, which describes the process of starting lessons with
things that are specific or tangible and then explicitly and gradually
transitioning to more abstract and general concepts.  Concreteness fading:

\begin{enumerate}

\item helps learners understand abstract symbols in terms of well-understood
  concrete objects,

\item lets them leverage personal experience to ground abstract thinking,

\item gives them a store of examples and mental images that they can fall back
  on when abstract symbols and reasoning fail, and

\item help learners figure out what is specific to particular examples and what
  is generalizable across all problems of a certain kind.

\end{enumerate}

One way to remember this strategy is the acronym PETE (Problem, Explanation,
Theory, Example), which encourages instructors to:

\begin{itemize}

\item describe an authentic problem that the lesson will solve,

\item work through a solution to that problem,

\item explain the general theory that underpins that solution, and

\item work through a second example so that learners will understand which parts
  generalize.

\end{itemize}

\rulemajor{Show How to Detect, Diagnose, and Correct Common Mistakes}

It is almost oxymoronic to say that learners spend a lot of their time trying to
figure out what they've done wrong and fixing it: after all, if they knew and
they had, they would already have moved on to the next subject.  Most lessons
devote little time to detecting, diagnosing, and correcting common mistakes, but
doing this will accelerate learning---not least by reducing the time that
learners spend feeling lost and frustrated.

In Carroll et al's ``minimal manual'' approach to training materials, every
topic is accompanied by descriptions of symptoms learners might see, their
causes, and how to correct them \cite{Carr2014}.  When studying second language
acquisition, \cite{Lyst1997} identified six ways in which instructors can
correct learners' mistakes:

\begin{description}

\item[Explicit correction:] clearly indicating that the learner is incorrect and
  provide the correct form.

\item[Recasting:] repeat the learner's response with the mistake or mistakes
  corrected.

\item[Clarification request:] indicate that the learner's answer is incorrect
  (e.g., by saying, ``Are you sure?'') but leave the correction open-ended.

\item[Metalinguistic clues:] pose leading questions (e.g., ``Do we need the
  absolute error or the relative error here?'')

\item[Elicitation:] provide the first part of the correct answer as a prompt and
  require the learner to fill in the rest.

\item[Repetition:] repeats the learner's error, drawing attention to it but
  leaving the correction up to them.

\end{description}

All of these can be used preemptively during the design of lessons.  For
example, an introduction to chemical reactions could present an incomplete
calculation of enthalpy and ask the learner to fill it in (elicitation) or
present the complete calculation with errors, then draw attention to those
errors and correct them one by one (recasting).  All of these strategies provide
retrieval practice by requiring learners to use what they have just learned, and
encourage metacognition by requiring them to reflect on the the limits and
applicability of that knowledge.

\rulemajor{Motivate and Avoid Demotivating}

One of the strongest predictors of whether people learn something is their
\emph{intrinsic motivation}, i.e., their innate desire to master the material.
The term is used in contract with \emph{extrinsic motivation}, which refers to
behavior driven by rewards such as money, fame, and grades.  As \cite{Wlod2017}
describes, the biggest motivators for adult learners are their sense of agency
(i.e., the degree to which they feel that they're in control of their lives),
the utility or usefulness of what they're learning, and whether their peers are
learning the same things.  Letting people go through lessons at the time of
their own choosing, using authentic tasks, and working in small groups speak to
each of these factors.

Conversely, it is very easy for educators to demotivate their learners by being
unpredictable, unfair, or indifferent.  If there is no reliable relationship
between effort and result, learners stop trying (a particular case of a broader
phenomenon called \emph{learned helplessness}).  If the learning environment is
slanted to advantage some people at the expense of others, everyone will do less
well on average \cite{Wilk2011}, and if the lessons make it clear that the
teacher doesn't care if people learn things or not, learners will mirror that
indifference.

One way to tell if learners are motivated or not is to look at the incidence of
cheating.  In classrooms, it is usually not a symptom of moral failing, but a
rational response to poorly-designed incentives.  As reported in
\cite{Lang2013}, some things that educators do that unintentionally encourage
cheating include:

\begin{itemize}

\item setting the cost of failure very high,

\item relying on single assessment mechanisms like multiple-choice tests, and

\item using arbitrary grading criteria.

\end{itemize}

Eliminating these from lessons doesn't guarantee that learners won't cheat, but
does reduce the incidence.  (And despite what many educators believe, cheating
is no more likely online than in person \cite{Beck2014}.)

\rulemajor{Make Lessons Inclusive}

\emph{Inclusivity} is a policy of including people who might otherwise be
excluded.  In STEM education, it means making a positive effort to be more
welcoming to women, under-represented racial or ethnic groups, people with
various sexual orientations, the elderly, the physically challenged, the
economically disadvantaged, and others.

One axis of inclusive lesson design is physical: provide descriptive text for
images and videos to help the visually challenged, closed captions for videos to
help those with hearing challenges, and so on.  Another axis is social:

\begin{itemize}

\item Use gender-neutral pronouns (e.g., a singular ``they'') or alternate
  between male and female pronouns.

\item Use culturally varied names in examples (e.g., Aisha and Boris rather than
  Alice and Bob).

\item Avoid examples based on over-simplified or exclusionary views of gender
  and orientation, such as assuming that there are only two genders, that gender
  is fixed throughout a person's life, or that marriage is always between people
  of unlike gender.

\end{itemize}

Committing fully to inclusive teaching may mean fundamentally rethinking
content.  For example, \cite{Lach2018} explored two strategies for making
computing education more culturally inclusive:

\begin{description}

\item[Community representation] highlights students' social identities,
  histories, and community networks using after-school mentors or role models
  from students' neighborhoods, or activities that use community narratives and
  histories as a foundation for a computing project.

\item[Computational integration] incorporates ideas from the learner's
  community, e.g., reverse engineering indigenous graphic designs in a visual
  programming environment.

\end{description}

The major risk of the community representation approach is shallowness, e.g.,
using computers to build slideshows rather than do any real computing.  The
major risk with computational integration is cultural appropriation, e.g., using
practices without acknowledging origins. When in doubt, ask your learners and
members of their community what they think you ought to do and give them control
over content and direction.

The most important step is to stop thinking in terms of a \emph{deficit model},
i.e., to stop thinking that the members of marginalized groups lack something
and are therefore responsible for not getting ahead. Believing that puts the
burden on people who already have to work harder because of the inequities they
face, and (not coincidentally) gives those who benefit from the current
arrangements an excuse not to look at themselves too closely.

\section*{Conclusion}

Following the ten rules laid out above doesn't guarantee that your lessons will
be great, but it will help ensure that they aren't bad.  When it comes time to
put them into practice, we recommend following something like the reverse design
process developed independently by \cite{Wigg2005,Bigg2011,Fink2013}:

\begin{enumerate}

\item Figure out who your learners are and what their goals are.

\item Create the summative assessment for the lesson to give yourself a target.

\item Itemize the knowledge and skills that assessment relies on, and create
  formative assessments to check on each while learning is taking place.

\item Order those formative assessments in a way that respects their
  dependencies, i.e., so that they build on each other.

\item Estimate the time required to cover each topic and perform its related
  formative assessment, then cut material that there isn't time for.

\item Write lessons to connect each formative assessment to the next (which is
  usually much easier than writing an entire lesson at once).

\item Double-check your language and examples to ensure that they address
  your learners' goals and won't demotivate them.

\item Derive learning objectives and key points from the lesson to share with
  your learners and co-instructors.  The former make the lesson findable, while
  the latter give you and your co-instructors a quick way to check what the
  lesson actually covers.

\item Put everything online for other people to download, modify, and contribute
  to.
  
\end{enumerate}

We also recommend that lessons be designed for sharing with other instructors.
Instructors often scour the web for ideas, and it's common for people to inherit
courses from previous instructors.  What is far less common is collaborative
lesson construction, i.e., people taking material, improving it, and then
offering their changes back to the community.  This model has served the open
source software community well, and as \cite{Deve2018} describes, it works
equally well for lessons---provided that materials are designed to make
fine-grained collaboration easy.

The best tool we have today for large-scale ad hoc collaboration is version
control.  Originally developed by programmers for managing the source code of
large projects, it gives each contributor their own working copy of the
material, but provides a way for them to submit changes to a central store,
have them reviewed, make further modifications, and finally merge them into
the core to strengthen it \cite{Blis2016}.

Version control's greatest strengths are its scalability (some projects have had
thousands of contributors) and its openness (anyone can offer changes).  Its
greatest weakness is that widely-used systems like Git are designed to handle
text files, and struggle with structured document formats like Microsoft Word or
PowerPoint.  As a result, people who wish to use it for lessons usually have to
rely on LateX or some variation of HTML or Markdown for their notes and slides,
which is a larger barrier to uptake than many experts realize (or are willing to
acknowledge).  If any reader wishes to be rich, famous, and popular (or at least
popular), building tools that enable version control tools to work with popular
office document formats would do more to help improve instructional quality than
any number of seminars on effective practices{\ldots}

Finally, one key enabler of collaborative lesson construction is licensing.  We
strongly recommend using one of the Creative Commons family of licenses, since
they have been carefully vetted and are widely understood.

\bibliography{rules}

\end{document}
